\newpage
\section{CSE 학생지원단 조직 구성 및 활동 기간}
\subsection{조직 구성 및 주요 활동}
CSE 학생지원단은 대외적인 업무를 담당하는 팀 리더 1명과,\\
학생지원단의 상시 업무를 담당하는 팀원으로 구성됩니다.\\
\\
팀 리더의 주요 업무는 학생지원단 외부의 인사들과 연락을 주고 받는 것입니다.\\
외부 인사에는 외부 연사, 교수님들, 행정실 선생님들과 교직원, 학생들이 포함됩니다.\\
팀 리더의 연락 업무 유형은 아래와 같습니다.
\begin{itemize}
    \item 교외 연락 업무: 외부 연사와의 섭외 연락 등의 업무 등.
    \item 교내 연락 업무: 학교 내 지도 교수님, 행정실 등 과의 연락 업무 등.
    \item 지원단 내 업무 조율: 지원단 부서 간 연락 사항 전달, 업무 조율 등.
\end{itemize}
\text{  }\\
팀원들의 주요 업무는 학생지원단의 상시적인 업무를 수행하는것 입니다.\\
팀원들의 업무 유형은 아래와 같습니다.
\begin{itemize}
    \item 소통 창구 관리: 소셜 미디어(Facebook, Instagram 등)을 통한 소통 창구 관리.
    \item 홍보물 제작: 행사 홍보물 제작, 행정실 홍보물 인가 획득 등
    \item 활동비 관리: 활동비 사용 내역 결산 관리 등
    \item 기타 업무: 행사 전후 지원, 설문조사 취합 및 분석, 활동 보고서 작성 등.
    \item 연락처 관리: 컴퓨터공학과에 소속된 전체 학생들의 이메일 리스트 최신화를 과 행정실에 요청하고 확인합니다.
\end{itemize}
\text{  }\\
팀 리더와 팀원들은 업무 상황을 실시간으로 공유하고,\\
필요한 경우 다른 팀원들의 업무를 지원할 수 있습니다.
\subsection{활동 기간 및 활동 시간}
\subsubsection{활동 기간}
학생지원단 활동 기간은 최소 한 학기(6개월) 입니다.\\
\\
6개월의 활동 기간이 끝난 후에 다음 학기까지 활동을 연장할지 스스로 결정할 수 있습니다. 학생지원단 활동 연장을 원하는 사람이 있을 경우, 활동 연장을 원하는 사람이 희망하는 역할군을 우선적으로 배정하고 추가 인력을 충원할 수 있습니다. \\
\\
활동기간 내에 활동을 중지 하기 위해서는 대체 인력을 직접 구해야 합니다.\\
대체 인력을 구하지 않고 활동을 중지할 수 있는 사유는 아래와 같습니다.
\begin{itemize}
    \item 군 휴학: 군 휴학을 신청하고 학기 중에 입영을 하였을 때.
    \item 질병 휴학: 학교에서 인정하는 사유로 인해 질병 휴학을 하였을때.
    \item 중도 제명: 부정행위, 태업, 횡령 등으로 인해 학생지원단에서 제명 되었을때.
\end{itemize}
\text{  }\\
\text{  }\\
활동 기간이 최소 기간에 미치지 못하는 경우, 학과에서는 활동 인증서를 발급하지 않으며 학생지원단 활동 사실을 기업체 등에 제출하는 서류에 기재할 수 없습니다.\\
\\
중도 제명되는 경우 제명 당시 활동 기간이 최소 기간을 상회하여도 활동 인증서를 발급 받을 수 없으며, 활동 사실을 기타 서류에 기재할 수 없습니다.\\

\subsubsection{활동 시간}
학생지원단은 학기 중에 매주 1회 전체 회의를 진행합니다.\\
전체 회의는 1시간 이내를 목표로 하여, 실무에 방해가 되지 않도록 합니다.\\
\\
정규 학기 중에 주 평균 활동 시간은 6시간 내외를 목표로 합니다.\\
방학 및 계절학기의 활동은 비대면으로 진행하며, 비정기적으로 운영할 수 있습니다.\\
\\
학생지원단이 개최하는 행사 진행 상황에 따라 활동시간을 유동적으로 늘리거나 줄일 수 있습니다. 활동 시간을 임의로 조정 할 때는 학생지원단을 담당하시는 컴퓨터공학과 홍보팀 교수님들께 미리 연락을 드리고 승인을 요청해야 합니다.\\
\\
학생지원단은 매주 활동 내역과 시간, 기타 논의 사항 등을 담은 활동 보고서를 지원단 전체 단위로 작성하여야 합니다. 작성한 활동 보고서는 특이 사항이 없는 경우 공개합니다.

\subsection{학생지원단 모집 개요}
\subsubsection{모집 시기}
봄 학기 및 가을 학기가 시작하는 첫 주에 모집 공고를 내야합니다.\\
\\
봄 학기와 가을 학기 첫 주에 모집 공고를 내기 위해서, 직전 여름 및 겨울 방학의 마지막 주 까지 모집 홍보물을 제작 완료 하고 지도 교수님의 승인을 받아야 합니다.\\
\\
모집 공고가 모든 학생들에게 전달될 수 있도록 과 행정실에 컴퓨터공학과 전체 메일 수신자 목록을 최신화 해야합니다.\\
\\
메일 수신자 목록을 미리 최신화 하여, 정규 학기가 시작되기 전에 전과를 한 사람들과 새내기학부에서 컴퓨터공학과로 새로 진입한 사람들을 모두 반영 해야합니다.\\

\subsubsection{모집 홍보물 공지 정보}
모집 홍보물에는 본 학생지원단 조직 개요를 바탕으로 아래 항목들을 고지합니다.\\
\begin{itemize}
    \item 모집 기간: 홍보물이 배포되기 시작한 날 부터 7일
    \item 모집 인원: 6인 내외로 상정.
    \item 모집 대상: 2학년 1학기 부터 4학년 1학기 재학중인 컴퓨터공학과 학생.\\
    ※ 복수전공자 또한 지원 가능.
    \item 활동 기간: 선발 완료 후 부터 다음 정규 학기 시작 전까지.
    \item 주 평균 활동 시간: 정규 학기 동안 6시간 이내, 방학 동안은 비대면 진행.
    \item 혜택: 활동 인증서, 회의비 지원, 우수 활동시 추천서 발행 등
\end{itemize}
\text{  }\\
홍보물은 국문과 영문 두 가지로 제작하여, 모든 학생들에게 배포될 수 있도록 합니다.\\

\subsubsection{모집 절차}

\begin{enumerate}
    \item \textbf{사전 준비}
    \begin{description}
        \item[메일 수신 대상자 최신화] \text{  }\\
        $\blacksquare$ 학사 일정이나 학교 포탈을 참조하여 전공 변경 기간이 마감되어 학생들의 전공 변경이 적용 된 것을 확인 한 이후, 과 행정실에 요청하여 컴퓨터공학과 전체 메일 수신 대상자를 최신화 합니다.\\
        $\blacksquare$ 필요시 최신화 된 명단을 직접 받는것을 검토할 수 있습니다.
        \item[활동 연장 희망자 확인] \text{  }\\
        이전 학기 활동자들 중 다음 학기에도 이어서 활동하고자 하는 사람이 있는 경우, 모집 인원을 산정할 때 참고 하기 위해서 미리 조사합니다.
    \end{description}
    \item \textbf{홍보물 제작 및 승인}
    \begin{description}
        \item[홍보물 제작] \text{  }\\
        $\blacksquare$ 2.3.2 의 "모집 홍보물 공지 정보" 를 과 행정실에 전달하여 홍보물 제작을 요청합니다.\\
        $\blacksquare$ 해당 정보를 영문으로도 같이 전달하여, 국문과 영문 홍보물을 같이 제작할 수 있도록 합니다.
        \item[홍보물 승인] \text{  }\\
        $\blacksquare$ 학생지원단 지도 교수님께 최종 홍보물을 승인 받아야 합니다.\\
        $\blacksquare$ 이후 과 행정실과 학사팀, 기숙사 행정실에서 각각 홍보물 승인 도장을 받아 학교 건물과 기숙사에 배포 할 수 있도록 승인을 받아야 합니다.
    \end{description}
    \item \textbf{모집 공고 배포}
    \begin{description}
        \item[학교 이메일] \text{  }\\
        $\blacksquare$ 최신화 된 명단을 바탕으로 모든 컴퓨터공학과 학생들에게 학교 이메일을 통해 배포합니다. 국문 이메일과 영문 이메일을 각각 따로 보냅니다.\\
        $\blacksquare$ 국제 학생들은 제목 영어로 작성되거나, 서두가 영어로 작성된 이메일을 우선적으로 읽기 때문입니다.
        \item[포스터 배포] \text{  }\\
        $\blacksquare$ 과 행정실, 학사팀, 기숙사 행정실에 각각 승인 도장을 받은 포스터들을 106동, 공학관 강의실, 기숙사 등에 배치합니다.
        \item[수업시간 홍보] \text{  }\\
        $\blacksquare$ 교수님들께 사전에 홍보물을 배포하여, 수업시간 전후에 컴퓨터공학과 학생지원단을 모집하고 있다는 사실을 알리는 것을 부탁드립니다.\\
        $\blacksquare$ 필요시 과 행정실을 통하여 관련 내용을 전달 할 수 있습니다.\\
        $\blacksquare$ 전공 필수 과목과 수강생이 많은 전공 선택 과목을 주요 대상으로 정합니다.\\
    \end{description}
    \item \textbf{최종 선발 및 발표}
    \begin{description}
        \item[지원자 개별 면접] \text{  }\\
        $\blacksquare$ 모든 지원자들은 학생지원단장과 짧은 개인 면담을 합니다.\\
        $\blacksquare$ 면담 주제는 활동 목표, 활동 의지, 관련 경험 등 입니다.
        \item[지원단장 면접] \text{  }\\
        $\blacksquare$ 신규 학생지원단장이 되기를 희망하는 사람은 학생지원단 지도교수님의 면접을 거쳐야 합니다.
        \item[선발 발표 및 활동 개시] \text{  }\\
        $\blacksquare$ 학생지원단의 모든 인원의 선발이 완료되면, 학생지원단 소통 창구 매체를 통해 선발이 완료 되었음을 발표합니다.\\
        $\blacksquare$ 선발이 완료됨과 동시에 학생지원단 학기 중 업무를 개시합니다.
    \end{description}
\end{enumerate}