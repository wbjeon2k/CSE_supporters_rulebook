\newpage
\section{CSE 학생지원단 조직 소개 및 조직 목표}

\subsection{CSE 학생지원단 조직 소개}
CSE 학생지원단 (이하 학생지원단) 은 컴퓨터공학과 학생 행사 기획과 추진을 담당하고, 컴퓨터공학과 학생과 학과의 소통을 원활하게 하기 위해서 설립 되었습니다.\\
\\
2021년 학과제가 본격적으로 도입 되어서 컴퓨터공학과가 기존의 전기전자컴퓨터공학부(ECE) 에서 분리 된 이후, 컴퓨터공학과의 자체적인 학생 조직이 필요성이 대두 되었고 이에 컴퓨터공학과 학과 차원에서 학생지원단을 조직 하였습니다.\\
\\
2022년에 신설된 학생지원단은 학과제 도입 이후 지속된 컴퓨터공학과 학생 조직의 공백을 해소합니다. 또한, 2020년 부터 2021년 까지 지속된 코로나 바이러스 대유행 방지를 위한 비대면 수업으로 인해 진행되지 않았던 학과 행사의 명맥을 다시 이어나가고, 장기적으로 운영하기 위해 노력 할 것입니다.

\subsection{CSE 학생지원단 조직 목표}
\subsubsection{학과 행사 개최}
CSE 학생지원단의 주 목표는 컴퓨터공학과 학생 행사의 지속적인 개최입니다.\\
\\
CSE 교수님들과 학생들이 함께 모이는 CSE Night 을 비롯한 \textbf{학과 내의 교류를 위한 행사}, 그리고 UNIST 프로그래밍 경진대회 등 \textbf{학술적인 행사}를 성공적으로 개최하고 운영 하는것을 추구합니다.\\
\\
주요 행사들을 지속가능한 형태로 정례화 하여, 풍부한 학과 행사들을 UNIST 컴퓨터공학과 학생 사회 문화로 정착 시키는 것을 목표로 삼고 있습니다.
\subsubsection{학과 학생들의 의견 전달}
CSE 학생지원단은 컴퓨터공학과 학생들의 의견을 수렴하여 학과측에 전달하고, 학과측의 요청사항을 학생들에게 효과적으로 전달하여 학과와 학생들 간의 지속적인 교류에 도움을 주는 역할을 추구합니다.\\
\\
기존의 총학생회와 정보바이오융합대학 학생회의 역할군과 중복되지 않는 선에서  학생지원단은 학생들의 의견을 학과 및 학교 기관에 전달하며 기존 단체들을 보조하는 역할을 맡고, 컴퓨터공학과 학생지원단의 고유의 역할을 우선시합니다.





