\newpage
\section{CSE 학생지원단이 운영하는 행사 목록}
\subsection{학술 행사}
\subsubsection{UNIST 프로그래밍 경진대회}
UNIST 프로그래밍 경진대회 (이하 Uni-CODE) 는 UNIST 학생들에게 알고리즘 문제해결 분야를 쉽게 접할 수 있도록 함으로써
진입장벽이 높은 알고리즘 문제해결 분야에 입문하고, 관련 커뮤니티를 형성 할 수 있도록 한다는 목표 아래에 개최되는 행사입니다.\\
\\
1학년 및 2학년 학생들에게는 알고리즘 문제풀이 경험을 접해볼 기회와 동기를 제공하여 추후 학과 공부에 도움을 주는 기대효과를 가지고 있습니다.\\
\\
3학년 및 4학년 학생들에게는 최근 알고리즘 문제해결 능력을 중요시하여 바뀌고 있는 기업 입사
코딩테스트를 대비할 수 있는 기회를 제공합니다.\\
\\
2019년 1회 행사가 개최 되었으며, 매년 11월에 개최 합니다.\\
2021년 제 3회 대회부터 UNIST 알고리즘 문제해결 동아리 \textbf{Almight} 에서 출제를 맡고 있습니다.

\subsection{학과 내 교류 행사}
\subsubsection{CSE Night}
CSE Night 은 UNIST 컴퓨터공학과 교수님들과 학생들이 같은 장소에 모여서 저녁 식사를 하며 진행하는 학과 내 교류 행사입니다.\\
\\
학과 수업 외의 교수와 학생들 간의 직접적인 접점이 부족하기에, CSE Night 을 통해서 학과 수업을 벗어난 자유로운 대화의 장을 마련하기 위한 행사입니다.\\
\\
학생들이 평소에 가진 관심사항이나 질문을 바탕으로 학술적인 질문과 개인적인 질문을 오가는 다양한 대화의 장을 마련하여 컴퓨터공학과 소개(UNI111) 과목 보다 더 자유로운 환경에서 자연스럽게 교류할 수 있습니다.\\
\\
또한 컴퓨터공학과 학생들이 학년과 상관없이 다양하게 만날 수 있는 상황을 만들어, 학과 학생들 간의 결속력을 같이 높이는 것을 부수적인 목표로 삼습니다.\\
\\
매 학기 종강 5주 전에 학생들의 학업 부담이 상대적으로 적고, 교수님들의 주요 학술지나 학회 마감일을 피하여 개최합니다.\\